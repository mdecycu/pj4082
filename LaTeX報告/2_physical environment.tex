\chapter{實體環境}

\section{3D列印機介紹}

\subsection{FIBR3DEmul}


 採用FIBR3DEmul,他的基層製造技術是將材料加熱後施加壓力噴嘴擠壓出來而定型,一層層的堆疊而形成三維的立體形狀。\\
 
 幾乎所有的基層製造系統一樣,擠製的基礎機器都是採用STL格式的CAD檔,零件的精度是由精度維持的,故輪廓會以較慢的速度製作,而輪廓是由從STL檔的平面和三角形之間提取的交點來做決定,這個方式最突出的優點是在任何想要的座標可以容易的切割。\\

 在成型設備上採用直角座標系統,其工作方式主要是通過完成沿著X、Y、Z軸上的線性運動,驅動單元是以伺服馬達或步進馬達為主,以線性滑軌或同步皮帶搭配齒輪或齒條作為傳動元件所架構起來的機械系統。\\

 原形的強度與填充率有關,強度減弱是孔隙率造成的,而孔洞大小是由填料密度去做選擇。\\

 擠製成型的技術結構中,擠製頭大多藉由切層軟體轉換機械指令GCode來執行點座標的材料塗層路徑,其中提升Z軸升降的穩定性、熔絲擠出量的控制及環境溫度差異變引起的材料熱收縮皆會影響疊層精度的表現,目前都是以手動調整彈簧機構來維持平台角度但此方法無法有效解決平台本身弧形的高低差,而G92是透過安裝於效應器上的現為開關偵測,藉由現為開關的觸發機制,以利於傳回平台與每個Z軸座標點的高度資料,當訊號傳回去時,立即儲存該點位置高度,並採線性補差方式來獲得完整的水平成型面。\\


