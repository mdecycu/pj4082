\renewcommand{\baselinestretch}{1.5} %設定行距
\pagenumbering{roman} %設定頁數為羅馬數字
\clearpage  %設定頁數開始編譯
\sectionef
\addcontentsline{toc}{chapter}{摘~~~要} %將摘要加入目錄
\begin{center}
\LARGE\textbf{摘~~要}\\
\end{center}
\begin{flushleft}
\fontsize{14pt}{16pt}\sectionef\hspace{12pt}\quad 自從1980年史上第一個RP技術專利的出現後,在時間的推進下,3D列印之相關技術不斷創新,從最初笨重又昂貴的3D列印機,漸漸地輕量化、普及化,價格也愈來愈實惠,時至今日,3D列印機已經可以應用於多種領域上。\\[12pt]

\fontsize{14pt}{16pt}\sectionef\hspace{12pt}\quad 此專題是使用虛擬建構的3D列印機,將其導入CoppeliaSim模擬環境並給予對應設置,並使用切片軟體對工件進行切片,轉出Gcode檔後,運用GCodeInterpreter導入Gcode檔與CoppeliaSim進行對接,操控CoppeliaSim場景中的三軸3D列印機進行模擬列印,可藉此展示與模擬實際列印的情況。\\[12pt]

\end{flushleft}
\fontsize{14pt}{16pt}\sectionef 關鍵字: CoppeliaSim、3D printer、FDM
\newpage
%=--------------------Abstract----------------------=%
\renewcommand{\baselinestretch}{1.5} %設定行距
\addcontentsline{toc}{chapter}{Abstract} %將摘要加入目錄
\begin{center}
\LARGE\textbf\sectionef{Abstract}\\

\begin{flushleft}
\fontsize{14pt}{16pt}\sectionef\hspace{12pt}\quad Since the emergence of the first RP technology patent in the history of 1980, under the promotion of time, 3D printing related technologies have been continuously innovated, from the initial bulky and expensive 3D printers, gradually lightweight, popular, and the price has become affordable, and today, 3D printers have been applied in a variety of fields.\\[12pt]

\fontsize{14pt}{16pt}\sectionef\hspace{12pt}\quad This topic is to use a virtually constructed 3D printer, import it into the CoppeliaSim simulation environment and give corresponding settings, and use the slicing software to transfer out the Gcode file, use GCodeInterpreter to import the Gcode file to dock with CoppeliaSim, and control the three-axis 3D printer in the CoppeliaSim scene for analog printing, which can show and simulate the actual printing situation.\\
\end{flushleft}
\end{center} 
\fontsize{14pt}{16pt}\sectionef Keyword: CoppeliaSim、3D printer、FDM 
