\chapter{前言}
\renewcommand{\baselinestretch}{10.0} %設定行距
\pagenumbering{arabic} %設定頁號阿拉伯數字
\setcounter{page}{1}  %設定頁數
\fontsize{14pt}{2.5pt}\sectionef
\section{研究動機}
近年來3D列印技術越來越普及,客製化以及利於處理複雜的形狀,使得3D列印在各個市場獨佔一席之地,但是列印需要耗費許多時間,因此我們放入虛擬環境模擬,可以有效率地看見成果。本專題以Uarm做為列印對象,將列印過程轉移到虛擬環境進行模擬,為了找到最符合效益的零件大小,所以我們將模型簡化後再放進各個軟體進行有限元素分析,透過模擬得到的數據進行比較產生最佳解。\\

\section{研究目的與方法}
第一部分使用CoppeliaSim進行虛擬環境模擬3D列印過程,第二部分是將Arduino與CoppeliaSim連結,第三部分是使用comsol及Ansys做有限元素分析。\\

簡化3D列印機,並使用XML格式導入虛擬環境,方便修改零件的坐標系及大小,加入GCode_interpreter及GCode,模擬3D列印。\\

建置CoppilaSim模擬環境,嘗試將2D訓練概念套用到3D環境進行測試,加入電腦視覺與RemoteAPI,電腦視覺抓取球與擊錘的位置,透過RemoteAPI進行遠端控制,在3D環境測試算法可確保後續套用到實體機器上的可行性。\\
 

\section{未來展望}

此次專題由構想到設計至模擬,模擬3D列印作為主軸深入到解決格式導入後產生的異位,並未了更擬真,在虛擬3D列印機中加入伺服馬達達到機電整合的設計,最後在進行模擬分析,找出其最佳化組合,希望未來可以進入到原形製造的部分。\\


\renewcommand{\baselinestretch}{0.5} %設定行距