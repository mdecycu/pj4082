\chapter{前言}
\renewcommand{\baselinestretch}{10.0} %設定行距
\pagenumbering{arabic} %設定頁號阿拉伯數字
\setcounter{page}{1}  %設定頁數
\fontsize{14pt}{2.5pt}\sectionef
\section{研究動機}
全球各種規模的公司都面臨著越來越快節奏、不確定和複雜的邊界條件,驅使這種現象的一個因素是數位化的日益增長迫使公司以更高的成本和時間效率進行開發。這次的專題藉由以客製化以及利於處理複雜的形狀的3D列印技術加入數位孿生創建即時的虛擬表示或在物理系統中實現以虛擬描述過程的概念,來達到目的。\\

\section{研究目的與方法}
第一部分,以CoppeliaSim作為虛擬環境,在環境中添加FIBR3DEuml,並將零件匯入Cura轉為GCode,接著將轉好的GCode放進GCodeInterpreter,模擬列印零件。\\

第二部分,利用將pySTL與cad檔做結合達到更精準的結果,將零件轉入虛擬環境模擬進行組配,再進行過程中如有發生錯誤,便可由上游的其他部分進行修改,可有效節省在實體列印上trial and error所浪費的時間。最終目的是透過如何利用虛擬環境提高準確性和擬真度以及減少時間、成本和工作量。\\

\renewcommand{\baselinestretch}{0.5} %設定行距